%%**************************************************************
%% Vorlage fuer Bachelorarbeiten (o.ä.) der DHBW
%%
%% Autor: Tobias Dreher, Yves Fischer
%% Datum: 06.07.2011
%%
%% Autor: Michael Gruben
%% Datum: 15.05.2013
%%**************************************************************

\input{ads/header}

% Ab jetzt können auch Umlaute verwendet werden

%falls pdftitel = titel der Arbeit
\newcommand{\titel}{\pdftitel}
%bei unterschiedlichen Titeln
%\newcommand{\titel}{I haben wir einen zweizeiligen
% Bachelorthesistitel}
\newcommand{\martrikelnr}{2929412}
\newcommand{\kurs}{WWI12B5}
\newcommand{\datumAbgabe}{Juli 2015}
\newcommand{\firma}{SAP SE}
\newcommand{\firmenort}{Walldorf}
\newcommand{\abgabeort}{Karlsruhe}
\newcommand{\abschluss}{Bachelor of Science}
\newcommand{\studiengang}{Wirtschaftsinformatik}
\newcommand{\dhbw}{Karlsruhe}
\newcommand{\betreuer}{Dr. Bernd Limberger}
\newcommand{\gutachter}{Torsten Rösen}
\newcommand{\zeitraum}{13.05.2015 bis 01.07.2015}
\newcommand{\arbeitsart}{\arbeit}

\makeglossaries
\input{ads/glossary}

\begin{document}



	% Deckblatt
	\begin{spacing}{1}
		\input{ads/deckblatt}
	\end{spacing}
	\newpage

	\renewcommand{\thepage}{\Roman{page}}
	\setcounter{page}{1}

	% Sperrvermerk
	\input{ads/sperrvermerk}
	\newpage
	
	% Erklärung
	%!TEX root = ../dokumentation.tex

\thispagestyle{empty}

\section*{Eidesstattliche Erklärung}
% http://www.se.dhbw-mannheim.de/fileadmin/ms/wi/dl_swm/dhbw-ma-wi-organisation-bewertung-bachelorarbeit-v2-00.pdf
\vspace*{2em}

Ich erkläre hiermit eidesstattlich, dass ich die vorliegende Arbeit selbstständig und ohne Benutzung anderer als der angegebenen Hilfsmittel angefertigt habe. Aus den benutzten Quellen direkt oder indirekt übernommene Gedanken habe ich als solche kenntlich gemacht.\\
\newline
Diese Arbeit wurde bisher in gleicher oder ähnlicher Form oder auszugsweise noch keiner anderen Prüfungsbehörde vorgelegt und auch nicht veröffentlicht.

\vspace{3em}

\abgabeort, \datumAbgabe
\vspace{4em}

\autor

	\newpage


	% Abstract
	%\input{ads/abstract}
	%\newpage

	\pagestyle{plain}

	% Inhaltsverzeichnis
	\begin{spacing}{1.1}
		\setcounter{tocdepth}{2}
		%für die Anzeige von Unterkapiteln im Inhaltsverzeichnis
		%\setcounter{tocdepth}{2}
		\tableofcontents
	\end{spacing}
	\newpage
	
		% Abkürzungsverzeichnis
		\cleardoublepage
		\phantomsection \label{listofacs}
		\addcontentsline{toc}{chapter}{Abkürzungsverzeichnis}
		%!TEX root = ../dokumentation.tex

\chapter*{Abkürzungsverzeichnis}
%nur verwendete Akronyme werden letztlich im Dokument angezeigt
\begin{acronym}[YTMMM]
\setlength{\itemsep}{-\parsep}

\end{acronym}

		
		
		% Abbildungsverzeichnis
		\cleardoublepage
		\phantomsection \label{listoffig}
		\addcontentsline{toc}{chapter}{Abbildungsverzeichnis}
		\listoffigures
	
		%Tabellenverzeichnis
		\cleardoublepage
		\phantomsection \label{listoftab}
		\addcontentsline{toc}{chapter}{Tabellenverzeichnis}
		\listoftables

    \cleardoublepage
	\renewcommand{\thepage}{\arabic{page}}
	\setcounter{page}{1}
	
	% Inhalt
     %!TEX root = ../dokumentation.tex
%\todo{Beteuer fragen wegen Zitierweise (Referenzen nach oder vor dem Punkt, wenn sich diese Referenz nicht nur auf einen Satz bezieht), Fußnoten von Acronyms, Schaubilder auf englisch, Schaubilder horizontal }

\chapter{Einführung}


     \input{content/02kapitel}
     \input{content/03kapitel}
     \input{content/04kapitel}
     
\chapter{Kritische Analyse des Leitfadens}

     \input{content/06kapitel}
     	
	
	% Anhang
	\clearpage
	\renewcommand{\thepage}{\Roman{page}}
		\setcounter{page}{8}
	%\pagenumbering{roman}
	\appendix
	\input{content/Anhang}
	



	% Quellcodeverzeichnis
	%\cleardoublepage
	%\phantomsection \label{listoflist}
	%\addcontentsline{toc}{chapter}{Listings}
	%\lstlistoflistings


	% Literaturverzeichnis
	\cleardoublepage
	\phantomsection \label{listoflit}
	\addcontentsline{toc}{chapter}{Literaturverzeichnis}
	
	\nocite{*}
	%Bib style
	%\bibliographystyle{agsm} %Havard  
	%\bibliographystyle{amsplain} %Durchnummeriert
	%\bibliographystyle{amsalpha} %Kürzel für Autor und Jahr,
	\bibliographystyle{geralpha} % was Jan verwendet: alpha
	%see more: http://amath.colorado.edu/documentation/LaTeX/reference/faq/bibstyles.pdf
	
  	\bibliography{ArbeitBib}
     
   %cd 
   \cleardoublepage
   \phantomsection
   \addcontentsline{toc}{chapter}{Beigabenverzeichnis}
   \chapter*{Beigabenverzeichnis}

\begin{figure}[H] 
  \centering
     \includegraphics[scale=0.35]{cd.pdf}
\end{figure}


Im Folgenden ist die Verzeichnisstruktur der beigelegten CD dargestellt.
 
\begin{description}
  \item[Bachelorarbeit]
    Beinhaltet die Bachelorarbeit als \textit{PDF}.
    Der zugehörige \emph{\LaTeX{}} Source Code befindet sich im Unterverzeichnis \textit{LaTeX}.
    Im Unterverzeichnis \textit{DOC} befindet sich die von \textit{PDF} in \textit{DOC} konvertierte Version dieser Arbeit.
  \item[Elektronische Quellen]
  %\todo{\BibTeX funktioniert nicht}
    Beinhaltet das Literaturverzeichnis im \textit{BibTeX} Format, sowie die digital verfügbaren Quellen als \textit{PDF} oder \textit{MHT}.
\end{description}

     
     
	% Glossar
	\printglossary[style=altlist,title=Glossar]
\end{document}
